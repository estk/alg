% ----------------------------------------------------------------
% AMS-LaTeX Paper ************************************************
% **** -----------------------------------------------------------
\documentclass{amsart}
\usepackage{graphicx}
\usepackage{amsfonts}
\usepackage{amscd}
\usepackage{amssymb}
\usepackage{mathtools}
% \usepackage{xy}
% ----------------------------------------------------------------
\vfuzz2pt % Don't report over-full v-boxes if over-edge is small
\hfuzz2pt % Don't report over-full h-boxes if over-edge is small
% THEOREMS -------------------------------------------------------
\newtheorem{thm}{Theorem}[section]
\newtheorem{cor}[thm]{Corollary}
\newtheorem{lem}[thm]{Lemma}
\newtheorem{prop}[thm]{Proposition}
\theoremstyle{definition}
\newtheorem{defn}[thm]{Definition}
\newtheorem{alg}[thm]{Algorithm}
\theoremstyle{remark}
\newtheorem{rem}[thm]{Remark}
\numberwithin{equation}{section}
% MATH -----------------------------------------------------------
\newcommand{\Matrix}[4]{ \left( \begin{array}{cc}  #1 & #2 \\  #3 & #4 \\ \end{array} \right) }
\newcommand{\norm}[1]{\left\Vert#1\right\Vert}
\newcommand{\abs}[1]{\left\vert#1\right\vert}
\newcommand{\set}[1]{\left\{#1\right\}}

\newcommand{\NN}{\mathbb N}
\newcommand{\ZZ}{\mathbb Z}
\newcommand{\QQ}{\mathbb Q}
\newcommand{\RR}{\mathbb R}
\newcommand{\CC}{\mathbb C}
\newcommand{\isom}{\cong}
\DeclarePairedDelimiter{\ceil}{\lceil}{\rceil}
\DeclarePairedDelimiter{\floor}{\lfloor}{\rfloor}

\pagestyle{plain}
% ----------------------------------------------------------------
\begin{document}
\title[]{Algorithms Homework 1}%
\author{Evan Simmons \\
        Dept. of Mathematics \& Computer Science \\ University of California Santa Cruz}%
%\date{}
%\dedicatory{}%
%\commby{}%
\renewcommand{\abstractname}{Homework Option}
% ----------------------------------------------------------------
\begin{abstract}
I would like to choose the homework heavy option.
\end{abstract}
\maketitle
% ----------------------------------------------------------------

\section{} Suppose we are given a sorted array of $n$ distinct elements
that has been circularly shifted by $k$ elements. We want to find an
algorithm to find the maximum that runs in $O(log(n))$ time.

\lem \label{lem1}

Consider two elements in our array, let us call them $A[i]$ and $A[j]$
where $i < j$. The elements between and including them are sorted if
and only if $A[i] < A[j]$. In other words: given any two elements in
a circularly sorted array, if they are in the correct order, then the
elements between them are in correct order.

\proof{} 

To prove the forward implication we will actually show the
contrapositive. Suppose the array slice $A[i..j]$ is not sorted, then
$\exists k \ni A[k] > A[k+1]$ where $i \leq k \leq j$. We know then
that since the array is circularly sorted that $\forall x \in A[i..k]$
and $\forall y \in A[(k+1)..j]$ we have $x > y$. Clearly $A[i] \in
A[i..k]$ and $A[j] y \in A[(k+1)..j]$, therefore $A[i] > A[j]$

To show the converse, we simply note that for any sorted array, the
following holds:

$$ \forall i,j \ni i<j \Rightarrow A[i] < A[j] $$

\cor \label{cor1}

Given a circularly sorted array and indices $i,j \in 0 \leq i < j \leq
n$, the maximum is between $i$ and $j$ if and only if $A[i] > A[j]$.

\proof 

By ~\ref{lem1} if 

\alg{(cirmax)}

If the array consists of a single element, return that element.

Let $h = \floor{n/2}$; compare the $A[0]$ and $A[h]$ elements. \\

Case (Less than): return the value of cirmax( $A[h..n]$ ) \\

Case (Greater than): return the value of cirmax( $A[0..(h-1)]$ ) \\

\proof

If an array consists of a single element, clearly that element must
be the maximum. If the first element $A[0]$ is less than the middle
element $A[h]$ then by ~\ref{cor1} the maximum occurs in the array slice
$A[h..n]$. However if $A[0]$ is greater than $A[h]$, then, again by
~\ref{cor1} the maximum occurs in $A[0..(h-1)]$ and we therefore only
need consider that slice.

\section{}


\subsection*{}

\end{document}
% ----------------------------------------------------------------
